\chapter{Conclusiones}
\graphicspath{{figs/}}
\label{Conclusiones}
\vspace*{-.1cm}

Se utilizaron herramientas estadísticas y computacionales para caracterizar frentes de onda gobernados por ecuaciones de reacción-difusión asociadas al modelo SIR sobre 
una variedad de medios estadísticamente isotrópicos. Estos frentes de onda son característicos de diversas fenomenologías asociadas a ecuaciones de reacción-difusión.

En particular, en el ámbito epidemiológico, modelar el efecto que tiene el carácter heterogéneo de la distribución espacial de las poblaciones sobre la dinámica 
infecciosa constituye un desafío complejo y sin lugar a dudas moderno.\cite{RILEY201568}

El trabajo desarrollado aquí constituye un paso en esa dirección. Fue posible 
obtener resultados no triviales respecto de la influencia del medio sobre los frentes de propagación. Características de interés como la velocidad de propagación, la 
amplitud media, la rugosidad del campo de desplazamiento e incluso la nocividad de los frentes fueron desarrolladas a nivel cuantitativo. Exponentes críticos asociados 
a un cambio radical de la dinámica fueron determinados sobre los distintos medios.

Por otro lado, en el modelado de incendios forestales es claro que la topografía, el carácter heterogéneo del medio y las condiciones climáticas son factores 
determinantes. Estos podrían introducirse en el modelo a partir de una interpretación precisa de los efectos que las variaciones en el medio tienen sobre la dinámica y 
ser estudiados utilizando las mismas herramientas desarrolladas aquí.

Es importante notar que exponentes críticos tal como los determinados en este trabajo, suelen estar asociados a una dada clase de universalidad, que resulta independiente 
de los detalles propios del problema estudiado aquí. \cite{barabasi1995fractal} Esto daría la posibilidad de estudiar sistemas más complejos dentro de la misma clase de universalidad a partir 
del modelo más simple que se ha explorado en este trabajo.

Para futuros trabajos en esta línea sería interesante investigar los efectos de la dinámica sobre medios hiper-uniformes, tanto ordenados como desordenados. Así como 
medios con configuraciones periódicas o bien tan arbitrarias como se quiera. También se propone trabajar con medios dinámicos y no solo estacionarios como los estudiados
aquí.
