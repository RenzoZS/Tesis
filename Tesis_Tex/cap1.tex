\chapter{Introducción}
\graphicspath{{figs/}}
\label{Intro}

El modelado matemático de una dada fenomenología constituye una herramienta fundamental en el proceso de entendimiento cuantitativo de la misma. Más aún, aplicado 
correctamente sobre una problemática concreta, como lo son las epidemias o los incendios forestales, permite desarrollar estrategias de contención, mitigación y 
prevención \cite{bressan2009existence}.

Entre la gran diversidad de desafíos que se presentan al momento de describir la dinámica de enfermedades infecciosas sobre una dada población o bien la propagación de un 
frente de incendio, se encuentra el desafío de representar correctamente el carácter heterogéneo de la distribución espacial de la población o vegetación \cite{RILEY201568}.
Esto, en última instancia, incluiría aspectos desde el ámbito comportamental de los individuos hasta la distribución espacial de los mismos. En tanto que para 
incendios forestales, involucra la topografía del terreno, la diversidad de vegetación y su distribución espacial y hasta contribuciones climáticas. 

El objetivo de este trabajo de tesis es dar un paso en esta dirección. Tanto para comprender los efectos que tienen sobre la dinámica las heterogeneidades del medio 
de sustentación, ya sea de la población o vegetación, como para desarrollar herramientas estadísticas y computacionales que puedan ser utilizadas en sistemas 
completamente diferentes donde la influencia de las características del medio sean de inteŕes. Para ello se consideró un modelo espacio-temporal de los más 
sencillos en lo que respecta a modelos epidemiológicos de tipo SIR (Susceptibles - Infectados - Recuperados) \cite{SIR,keeling:infectious_diseases,Noble1974GeographicAT,kolton},
en donde la movilidad de los individuos es dominada por un término difusivo\cite{chuleta}, tal como se verá en detalle en el capítulo \ref{Modelo teórico}. 

Por su parte, las heterogeneidades del medio se introdujeron por medio de la distribución espacial de la tasa de transmisión la cual ha mostrado tener implicaciones 
significativas para reproducir patrones de propagación espacio-temporales dados por datos epidemiológicos \cite{mosquito}.
\newpage

El presente trabajo se divide en tres capítulos además del presente, los cuales se describen brevemente a continuación:

En el \textbf{Capítulo 2} se presenta el marco teórico del trabajo, las herramientas estadísticas utilizadas para caracterizar los frentes de onda y se precisa las 
condiciones del problema a resolver junto con la caracterización de los distintos medios que se propone estudiar.

En el \textbf{Capítulo 3} se presentan los resultados obtenidos a partir de las simulaciones numéricas realizadas masivamente para cubrir diferentes parámetros del 
problema y fundamentalmente para cuantificar los efectos sobre la dinámica de los distintos medios de sustentación.

En el \textbf{Capítulo 4} se comentan brevemente las conclusiones del trabajo, sus potenciales aplicaciones y un posible desarrollo a futuro.





