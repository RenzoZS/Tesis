\begin{resumen}
    Se estudió la propagación de frentes de onda gobernados por ecuaciones de reacción-difusión en el marco del modelo SIR espacial.
    Dichos frentes de onda podrían utilizarse para caracterizar frentes de infección en una problemática epidemiológica o bien orientarse a una problemática completamente
    diferente como lo son los frentes de incendios. Se definió una metodología estadística para la caracterización de los frentes de onda a partir de la cual se obtuvieron
    resultados cuantitativos respecto de la velocidad, la amplitud media, las propiedades geométricas e incluso la nocividad de los frentes sobre diferentes medios
    isotrópicos. En particular, se exploraron medios homogéneos, desordenados y correlacionados a partir de lo cual pudo describirse cuantitativamente el efecto que tenía
    cada uno de ellos sobre las características del frente de onda.
     
    Se realizaron simulaciones numéricas masivas para resolver el sistema de ecuaciones de reacción-difusión involucrado en la dinámica. Estas se implementaron
    de manera eficiente utilizando computación acelerada a través de programación en paralelo sobre procesadores gráficos. De esta manera fue posible obtener resultados
    sobre sistemas a gran escala en tiempos razonables.
\end{resumen}

\begin{abstract}
    The propagation of wave fronts governed by reaction-diffusion equations were studied within the framework of the spatial SIR model.
    These wave fronts could be used to characterize infection fronts in an epidemiological problem or be oriented to a completely different problem such as fire fronts.
    A statistical methodology was defined for the characterization of the wave fronts from which quantitative results were obtained regarding the speed, the mean 
    amplitude, the geometric properties and even the harmfulness of the fronts on different isotropic media. In particular, homogeneous, disordered and correlated 
    media were explored, from which it was possible to quantitatively describe the effect that each of them had on the characteristics of the wavefront.
    
    Massive numerical simulations were performed to solve the system of reaction-diffusion equations involved in the dynamics. These were efficiently implemented using 
    accelerated computing through parallel programming on graphics processors. In this way it was possible to obtain results on large-scale systems in reasonable times.    
\end{abstract}
