\begin{resumen}
    Se estudió la propagación de frentes de onda gobernados por ecuaciones de reacción-difusión en el marco del modelo SIR espacial.
    Dichos frentes de onda podrían utilizarse para caracterizar frentes de infección en una problemática epidemiológica o bien orientarse a una problemática completamente
    diferente como la de frentes de incendios. Se definió una metodología estadística para la caracterización de los frentes de onda a partir de la cual se obtuvieron
    resultados cuantitativos respecto de la velocidad, la amplitud media, las propiedades geométricas e incluso la nocividad de los frentes sobre diferentes medios homogéneos y heterogéneos. La heterogeneidad se introdujo en el modelo por medio de una tasa de transmisión espacialmente dependiente. En particular, se exploraron heterogeneidades desordenadas y correlacionadas a partir de lo cual pudo describirse cuantitativamente el efecto que tenía
    cada uno de ellas sobre las características del frente de infección.
     
    Se realizaron simulaciones numéricas masivas para resolver el sistema de ecuaciones de reacción-difusión involucrado en la dinámica. Estas se implementaron
    de manera eficiente utilizando computación acelerada a través de programación en paralelo sobre procesadores gráficos. De esta manera fue posible obtener resultados
    sobre sistemas a gran escala en tiempos razonables.

    Encontramos una dependencia no trivial del umbral de propagación y la velocidad del frente con el desorden y observamos diferencias dependiendo del tipo de desorden involucrado. Estudiamos la dinámica de la rugosidad del frente de infección y determinamos que la misma cae en la clase de universalidad de KPZ (Kardar-Parisi-Zhang) de origen cinético.
\end{resumen}

\begin{abstract}
    The propagation of wave fronts governed by reaction-diffusion equations was studied within the framework of the spatial SIR model.
    These wave fronts could be used to characterize infection fronts in an epidemiological problem or be oriented to a completely different problem such as forest 
    fire fronts. A statistical methodology was defined for the characterization of the wave fronts from which quantitative results were obtained regarding the speed, the mean amplitude, the geometric properties and even the harmfulness of the fronts on different homogeneous and heterogeneous media. Heterogeneity was introduced into the model by means of a spatially dependent transmission rate. In particular, disordered and correlated heterogeneities were explored, from which it was possible to quantitatively describe the effect that each of them had on the characteristics of the wavefront.
    
    Massive numerical simulations were performed to solve the system of reaction-diffusion equations involved in the dynamics. These were efficiently implemented using 
    accelerated computing through parallel programming on graphics processors. In this way it was possible to obtain results on large-scale systems in reasonable times.
    
    We find a non-trivial dependence of the propagation threshold and the front velocity with disorder and observe differences depending on the type of disorder involved. We study the dynamics of the roughness of the front and determine that it could be explained by the KPZ (Kardar-Parisi-Zhang) universality class of kinetic origin.
\end{abstract}
