\chap{Conclusiones}{conclu}
\graphicspath{{figs/cap6}}


Comenzamos este trabajo (Capítulo \ref{cap:intro}) con una investigación en donde observamos, a modo general, la relevancia física que tienen las inferfases fuera del equilibrio. Mostramos una diversidad de ejemplos en la naturaleza donde la dinámica de un sistema queda caracterizada por una interfase. Todos ellos ampliamente estudiados en la literatura, ya sea desde los frentes de incendio o epidemiológicos hasta las olas producidas por las \textit{abejas gigantes} y el crecimiento de bacterias \cite{zhang1992modeling, provatas1995scaling, PhysRevLett.79.1515, Jullien1992SurfaceD, PhysRevLett.110.035501, kastberger2014speeding, kastberger2013social, kastberger2008social, matsushita1990diffusion, bhattacharjee2022chemotactic, mate_sist_bio, barbieri2020soil, stenseth2008plague}. Luego, se describieron las herramientas necesarias y discutimos conceptos de escaleo para poder cuantificar y describir precisamente las características fundamentales de las interfases.





La completitud de los resultados obtenidos en este trabajo se obtuvieron utilizando una librería de \textit{Python} desarrollada personalmente que se encuentra accesible en un repositorio de \textit{GitHub} (\href{https://github.com/RenzoZS/tesis}{https://github.com/RenzoZS/tesis})\footnote{Ojalá se encuentre lo más documentada posible al momento de leer esto.}. La misma precisa únicamente del paquete \textit{CuPy} y una GPU compatible con \textit{CUDA-toolkit} para poder ejecutarse, además de \textit{Python} claro. La metodología numérica implementada se encuentra discutida al modo más general, autocontenido y preciso que pude en el Capítulo \ref{cap:code}. Donde también se muestra la ventaja de usar una GPU en la resolución de un sistema de ecuaciones diferenciales parciales con sistemas de gran tamaño.




