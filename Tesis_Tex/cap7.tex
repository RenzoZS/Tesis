\chap{Conclusiones}{conclu}

Comenzamos este trabajo (Capítulo \ref{cap:intro}) con una investigación en donde observamos, a modo general, la relevancia física que tienen las inferfases fuera del equilibrio. Mostramos una diversidad de ejemplos en la naturaleza donde la dinámica de un sistema queda caracterizada por una interfaz. Todos ellos ampliamente estudiados en la literatura, ya sea desde los frentes de incendio o epidemiológicos hasta las olas producidas por las \textit{abejas gigantes} y el crecimiento de bacterias \cite{zhang1992modeling, provatas1995scaling, PhysRevLett.79.1515, Jullien1992SurfaceD, PhysRevLett.110.035501, kastberger2014speeding, kastberger2013social, kastberger2008social, matsushita1990diffusion, bhattacharjee2022chemotactic, mate_sist_bio, barbieri2020soil, stenseth2008plague}. Luego, se describieron las herramientas necesarias y discutimos conceptos de escaleo para poder cuantificar y describir precisamente las características fundamentales de las interfaces.

Pasamos de la visión general a un problema particular (Capítulo \ref{cap:SIR}), orientado a describir la propagación de una epidemia en un medio heterogéneo, sustentado sobre un modelo de tipo SIR espacial con difusión (ecuaciones \ref{Seqfinal} - \ref{Ieqfinal}), en donde introdujimos heterogeneidad por medio de una tasa de transmisión espacialmente dependiente. Inicialmente, discutimos las soluciones de onda solitaria asociadas a este modelo en el caso homogéneo, con tasa de transmisión constante en el espacio, y determinamos la velocidad del frente de infección (ecuación \ref{eq:c0}), el umbral de propagación dado por una tasa de transmisión crítica (ecuación \ref{eq:beta_c_v}) y los perfiles asintóticos del mismo (ecuaciones \ref{larika2} - \ref{larika3}).

Luego, nos dispusimos a estudiar el sistema con medios heterogéneos, con el objetivo de determinar la influencia de las heterogeneidades sobre el frente de infección. Primero investigamos los observables macroscópicos (Capítulo \ref{cap:resultados_criticos}), en particular estudiamos la velocidad (\ref{velocidad}), la amplitud máxima (\ref{maximo}) y la nocividad (\ref{S1}) del frente de infección en función de la tasa de transmisión media espacial $\beta_m$ utilizando distintas heterogeneidades. Encontramos fenómenos críticos para todas ellas asociadas al umbral de propagación del frente y determinamos los exponentes críticos asociados. Observamos un fenómeno no trivial y emergente de aceleración del frente al usar medios heterogéneos respecto de medios homogéneos. Más aún, observamos que el umbral de propagación es sensible a la heterogeneidad, dando lugar a la propagación más fácilmente en medios heterogéneos que en medios homogéneos. Es de relevancia notar una interpretación pragmática de estos resultados: imagine que se quiere evitar una epidemia reduciendo la tasa de transmisión por vacunación o aislamiento social, sería muy productivo saber cómo depende el umbral de propagación con la distribución espacial de la tasa de transmisión para diseñar una estrategia espacial de vacunación o aislamiento eficiente. De manera similar, si se quiere evitar la propagación de incendios utilizando cortafuegos podría diseñarse una buena estrategia para la ubicación espacial de los mismos.\footnote{Aclaro que no estamos diciendo que los resultados obtenidos con este modelo en particular sean útiles para casos reales, solo es para dar una idea cualitativa de la importancia.}

Finalmente, nos centramos en estudiar propiedades microscópicas y universales del frente de infección con la heterogeneidad DA (Capítulo \ref{cap:rugosidad}). Determinamos los exponentes de rugosidad y dinámico de la interfaz, los cuales definen la clase de universalidad del frente. Encontramos que el exponente dinámico coincide con el exponente dado por la clase de universalidad de KPZ. Por otro lado, verificamos la existencia de una contribución no lineal a la dinámica de la interfaz viendo si había alguna dependencia de la velocidad del frente con la inclinación del mismo. Más aún, encontramos que esta contribución es de origen cinético y que es proporcional a la velocidad del frente. Sin embargo, el exponente de rugosidad requiere de estudios más precisos dado que aparece ligeramente subestimado.

De todas formas, es importante notar que la dinámica rugosa del campo de desplazamiento que define la interfaz de un frente de infección en un modelo SIR como el discutido aquí presenta características similares a las descritas por la universalidad de KPZ con origen cinético. Recordemos que la ecuación de KPZ es esencialmente estocástica mientras que el modelo SIR de difusión discutido en este trabajo es determinista: dada la misma tasa de transmisión espacial y las mismas condiciones iniciales, el sistema evolucionara siempre de la misma manera. Es decir, \textbf{el comportamiento universal emerge al observar la dinámica del frente de forma estadística sobre una variedad de tasas de transmisión $\beta_{\vb{r}}$ generadas aleatoriamente.}

A futuro sería interesante intentar derivar una ecuación para el frente $u(y,t)$ utilizando técnicas como las que se usaron en Ref. \cite{provatas1995scaling}. Por otro lado, sería importante verificar si utilizando una interfaz de mayor tamaño se obtiene un exponente de rugosidad distinto y más cercano al de KPZ. Se podría intentar hacer una caracterización más exhaustiva de los exponentes de universalidad en función del parámetro de desorden $p$ para el medio DA y estudiarlos también en con otras heterogeneidades. Por último, sería interesante también ver qué efecto tiene sobre la rugosidad de la interfaz incluir difusión en la ecuación de susceptibles, es decir, $D_{S} \neq 0$, dado que como comentamos en el Capítulo \ref{cap:SIR} esto afecta a la estabilidad del mismo y podría producir alteraciones no triviales en la interfaz.

Por otro lado, lo más atractivo probablemente sea extender las herramientas desarrolladas aquí para estudiar diversos modelos físicos, químicos, ecológicos, etc. que estén definidos con ecuaciones de reacción difusión y presenten algún tipo de heterogeneidad. Pueden encontrarse varios trabajos recientes al respecto, algunos de ellos son \textit{Front roughening of flames in discrete media} \cite{PhysRevE.96.013107} y \textit{Fisher waves and front roughening in a two-species invasion model with preemptive competition} \cite{PhysRevE.74.041116}.

La completitud de los resultados obtenidos en este trabajo se obtuvieron utilizando una librería de \textit{Python} desarrollada personalmente que se encuentra accesible en un repositorio abierto de \textit{GitHub} (\href{https://github.com/RenzoZS/tesis}{https://github.com/RenzoZS/tesis})\footnote{Espero que se encuentre lo más documentado posible al momento de leer esto.}. La misma precisa únicamente del paquete \textit{CuPy} y una GPU compatible con \textit{CUDA-toolkit} para poder ejecutarse, además de \textit{Python} claro.




