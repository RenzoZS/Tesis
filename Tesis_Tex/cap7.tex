\chap{Conclusiones}{conclu}
\graphicspath{{figs/cap6}}


Comenzamos este trabajo (Capítulo \ref{cap:intro}) con una investigación en donde observamos, a modo general, la relevancia física que tienen las inferfases fuera del equilibrio. Mostramos una diversidad de ejemplos en la naturaleza donde la dinámica de un sistema queda caracterizada por una interfase. Todos ellos ampliamente estudiados en la literatura, ya sea desde los frentes de incendio o epidemiológicos hasta las olas producidas por las \textit{abejas gigantes} y el crecimiento de bacterias \cite{zhang1992modeling, provatas1995scaling, PhysRevLett.79.1515, Jullien1992SurfaceD, PhysRevLett.110.035501, kastberger2014speeding, kastberger2013social, kastberger2008social, matsushita1990diffusion, bhattacharjee2022chemotactic, mate_sist_bio, barbieri2020soil, stenseth2008plague}. Luego, se describieron las herramientas necesarias y conceptos de escala para poder cuantificar y describir precisamente las características fundamentales de las interfases.








